\documentclass[]{article}

% insert here the call for the packages your document requires
\usepackage[utf8]{inputenc}
% Use nameref to cite supporting information files (see Supporting Information section for more info)
\usepackage{nameref,hyperref}
% amsmath and amssymb packages, useful for mathematical formulas and symbols
\usepackage{amsmath,amssymb}
% useful for consistent display and control of units of measurement
\usepackage{siunitx}
% figures!
\usepackage{graphicx}
% for including TODO notes
\usepackage{todonotes}
% syntax highlighting for code
\usepackage{minted}
% Uncomment next two lines to add line numbers.
\usepackage{lineno}
\linenumbers

\begin{document}

\title{Supplementary Materials for Analysis and Ethical Design of Terrain Park Jumps for Snow Sports}

\author{
  Jason~K. Moore \and
  Bryn Cloud \and
  Mont Hubbard \and
  Christopher~A. Brown
}

\maketitle

\section{Example Software Library Use}
\label{sec:example}
%
The closed form equation
%
\begin{align}
  h = \left[\frac{x^2}{4(x\tan\theta_T - y)\cos^{2}\theta_T} - y\right]
    \sin^{2}
    \left[\tan^{-1}\left(\frac{2y}{x} - \tan\theta_T\right) -
    \tan^{-1}\frac{dy}{dx}\right]
  \label{eq:efh}
\end{align}

is useful for understanding the
fundamental relationship of equivalent fall height to the landing surface
shape. It will predict EFH for small jumps but other factors may be useful to
include in the model. For example, jumpers are subject to aerodynamic drag and
this is not negligible for larger jumps. If drag is included there is no closed
form solution for the equivalent fall height, but the equivalent fall height
can be computed through iterative simulation~\cite{Levy2015}. The jumper's
flight path is found by integrating the flight equations of motion at various
takeoff velocities and computing the misalignment of jumper landing and slope
angles to then compute the equivalent fall height. This more general simulation
method is implemented in the software described herein and the results reflect
the inclusion of both gravitational and drag forces. Even with drag
incorporated, the calculating EFH still only require measurements of the
landing surface cross-sectional profile coordinates $(x,y)$ relative to the
takeoff point and a measurement of the takeoff angle.
Listing~\ref{lis:example-efh-calc} demonstrates the new software library
features creating a surface from some measured data points and then calculating
the equivalent fall height at 0.2\si{\meter} increments.
%
\begin{listing*}
  \begin{minted}{pycon}
>>> import numpy as np
>>> from skijumpdesign import Surface, Skier, plot_efh
>>> takeoff_ang = 10  # degrees
>>> takeoff_point = (0, 0)  # (x, y) in meters
>>> x_ft = np.array([-232.3,-203.7,-175.0,-146.3,-117.0,-107.4,
...    -97.7,-88.0,-78.2,-68.5,-58.8,-49.1,-39.4,-34.5,-29.7,
...    ...
...    38.8,43.3,47.8,52.3,56.8,61.5,66.2,70.9,75.7,80.6,85.5,
...    88.4,88.4])
...
>>> y_ft = np.array([55.5,46.4,37.7,29.1,22.2,19.7,17.2,14.8,
...    12.5,10.2,7.7,5.2,2.9,1.8,0.7,-0.2,-1.0,-1.2,-1.4,-1.6,
...    ...
...    -16.2,-18.1,-19.8,-21.4,-22.9,-24.0,-25.0,-25.6,-25.6])
...
>>> x_mt = x_ft*0.3048 # convert to meters
>>> y_mt = y_ft*0.3048 # convert to meters
>>> # create a surface from the data
>>> measured_surf = Surface(x_mt, y_mt)
>>> # create a skier
>>> skier = Skier(mass=75.0, area=0.34, drag_coeff=0.821)
>>> # calculate the equivalent fall height
>>> x, efh, v = measured_surf.calculate_efh(
...     np.deg2rad(takeoff_ang), takeoff_point, skier, increment=0.2)
...
>>> x  # display the x coordinates
array([ 0. ,  0.2,  0.4,  0.6,  0.8,  1. ,  1.2,  1.4,  1.6,  1.8,  2. ,
        2.2,  2.4,  2.6,  2.8,  3. ,  3.2,  3.4,  3.6,  3.8,  4. ,  4.2,
       ...
       24.2, 24.4, 24.6, 24.8, 25. , 25.2, 25.4, 25.6, 25.8, 26. , 26.2,
       26.4, 26.6, 26.8])
>>> efh  # display the equivalent fall height for each x coordinate
array([0.        , 0.02541035, 0.03479384, 0.03264587, 0.05956476,
       0.09096091, 0.12358184, 0.13702364, 0.15202999, 0.17018343,
       ...
       3.93910556, 3.97387212, 4.00891899, 4.04424779, 4.07984952,
       4.11573359, 4.68049185, 5.53413479, 6.45253722, 7.42628019])

>>> v  # display takeoff speeds to reach x positions
array([0.07373847, 0.13081777, 0.1878382 , 0.2447865 , 0.30166299,
       0.35851949, 0.41537661, 0.47221055, 0.52897197, 0.58564902,
       ...
       6.71699974, 6.76760188, 6.81816819, 6.86869777, 6.9191902 ,
       6.96962124, 7.02001551, 7.07037288, 7.1206941 ])
>>> # calculate and plot the efh curve
>>> plot_efh(measured_surf, takeoff_ang, takeoff_point, increment=0.2)
  \end{minted}
  \caption{Python interpreter session illustrating how one could compute the
  equivalent fall height of a measured jump.}
  \label{lis:example-efh-calc}
\end{listing*}

\section{Jump Shape Measurement}
\label{sec:jump-shape-measurement}
%
Calculating equivalent fall height requires the Cartesian coordinates and slope
of the landing surface along the path of the jumper. There are a number of
possible measurement techniques for collecting data adequate for the equivalent
fall height calculation but the simplest method requires only a digital level
\footnote{Smartphone digital level measurement applications are likely
sufficient and readily available.}, a flexible tape measure, and less than an
hour's time from one person per jump. A tenth of a degree accuracy from the
level and down to 25~\si{\centi\meter} accuracy from the tape measure should be
more than sufficient for typical snowsport jumps.

To measure the jump, the takeoff point should be identified and the tape
measure should then be draped over the contour of the landing surface along the
projection of the expected flight path onto the landing surface. The origin of
the tape measure should be aligned with the takeoff point. Starting with the
takeoff point, the digital level should be used to record the absolute angle at
regular increments along the tape. The increment can be varied between
25~\si{\centi\meter} and 100~\si{\centi\meter}, with the former used for steep
slope changes and the later for less steep; 50~\si{\centi\meter} increments are
appropriate for average jump shapes. Positive angles should be recorded for
positive slope and negative angles for negative slope. The tabulated data
should include the distance along the surface from the takeoff point, $d_i$,
and the associated surface angle, $\theta_i$, at each distance measurement for
$n$ measurements. Assuming $\theta_i$ is in radians, the Cartesian coordinates
can be computed using the average angle to find the adjacent coordinates. The
following equations show the calculation of the Cartesian coordinates from
these two measures used in the software.
%
\begin{align}
  \frac{dy_i}{dx_i} & = \tan^{-1}{\theta_i} \quad \text{for } i=1\ldots n \\
  x_{i + 1} & =
  \begin{cases}
    0 & \text{for } i=0 \\
    x_i + (d_{i+1} - d_i)\cos{\frac{\theta_{i+1} + \theta_i}{2}} &  \text{for }
    i=1\ldots n-1
  \end{cases} \\
  y_{i + 1} & =
  \begin{cases}
    0 & \text{for } i=0 \\
    y_i + (d_{i+1} - d_i)\sin{\frac{\theta_{i+1} + \theta_i}{2}} &  \text{for }
    i=1\ldots n-1
  \end{cases}
\end{align}

Listing~\ref{lis:example-meas-calc} demonstrates calculating the landing
surface's Cartesian coordinates from measured distance and angle data collected
with the method described above.
%
\begin{listing*}
  \begin{minted}{pycon}
>>> import numpy as np
>>> from skijumpdesign import cartesian_from_measurements
>>> dis = np.array([14.5, 15.0, 15.5, 16.0, 16.5, 17.0])  # meters
>>> ang = np.deg2rad([4.6, -7.4, -16.5, -9.7, -11, -6.9])  # radians
>>> x, y, to_point, to_angle = cartesian_from_measurements(dis, ang)
>>> print(x)  # meters
[0.         0.49985074 0.98901508 1.47600306 1.96786738 2.46177962]
>>> print(y)  # meters
[ 0.         -0.01221609 -0.1157451  -0.22907075 -0.31890113 -0.39668737]
>>> print(to_point)  # takeoff point in meters
(0.0, 0.0)
>>> print(to_angle)  # takeoff angle in radians
0.08028514559173916
  \end{minted}
  \caption{Python interpreter session showing how one could compute the
  Cartesian coordinates from equivalent fall height of a measured jump.}
  \label{lis:example-meas-calc}
\end{listing*}

\bibliographystyle{ieeetr}
\bibliography{references}   % name your BibTeX data base

\end{document}
