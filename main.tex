\documentclass{article}

\usepackage[utf8]{inputenc}
% Use nameref to cite supporting information files (see Supporting Information section for more info)
\usepackage{nameref,hyperref}
% amsmath and amssymb packages, useful for mathematical formulas and symbols
\usepackage{amsmath,amssymb}
% useful for consistent display and control of units of measurement
\usepackage{siunitx}
% figures!
\usepackage{graphicx}
% for including TODO notes
\usepackage{todonotes}
% Uncomment next two lines to add line numbers.
\usepackage{lineno}
\linenumbers

\begin{document}
\title{Online Software and Ethical Issues for Safety-Conscious Design of Terrain Park Jumps}

\author{
  Jason~K. Moore \and
  Bryn Cloud \and
  Mont Hubbard \and
  Christopher~A. Brown \\\\
  J.~K. Moore
  Delft University of Technology\\
  Mekelweg 2, 2628 CD Delft, The Netherlands\\
  j.k.moore@tudelft.nl
  \and
  B. Cloud \& M. Hubbard
  University of California, Davis\\
  One Shields Ave., Davis, CA 95616 USA\\
  becloud@ucdavis.edu,mhubbard@ucdavis.edu
  \and
  C.~A. Brown
  Worcester Polytechnic Institute\\
  100 Institute Rd., Worcester, MA 01609 USA\\
  brown@wpi.edu
}

\maketitle

\begin{abstract}
  Many snowsport resorts now have terrain parks and decades-long
  epidemiological evidence correlates terrain park use with injuries.
  Engineering design of jumps could reduce injuries by limiting equivalent fall
  heights, which are proportional to dissipated landing impact energy.  No
  evidence refutes making terrain park jumps safer in this way. We discuss case
  studies illustrating that large equivalent fall heights are significant
  factors in traumatic injuries on terrain park jumps. We argue that it is the
  ethical responsibility of engineers to ensure the safety, health, and welfare
  of the public when performing and presenting research on snowsport safety.
  Developing standards and adopting design tools for builders can make jumps
  safer. To reduce injuries, we introduce an online tool that can evaluate
  existing jumps as well as design jump profiles with safer equivalent fall
  heights.
\end{abstract}

\section*{Keywords}

skiing; snowboarding; safety; ethics

\section{Introduction}
\label{intro}
%
Impacts with fixed surfaces can cause injury. Greater velocities, perpendicular
to the surfaces, provide greater injury potential due to increased kinetic
energy dissipation. Equivalent fall height (EFH) is a conceptually simple and
familiar measure of impact danger used in safety standards worldwide, from
construction~\cite{OSHA2021} to children's playground
equipment~\cite{Chalmers1996}. EFHs of terrain park jumps can be calculated
using techniques in \cite{Levy2015} from Cartesian coordinates of jump
profiles. These coordinates include starting points, takeoff ramps, and landing
hills, all along jumpers' paths. Limiting energy dissipation in human bodies,
hence EFH on jumps, reduces likelihoods of injuries and their severities. EFH
should be considered because it is clearly connected to injury risk and can be
used to design and construct safer jumps. In fact, safety
research~\cite{Smith2020} tells us that designing forgiving environments (i.e.,
limiting EFH at all possible landing locations) is more effective than forcing
behavioral change (e.g., requiring the jumper to regulate their speed to ensure
a landing only in a small safe region).

Societal costs of jump injuries are discussed here with case studies that
illustrate dangers if EFH is not limited appropriately. We also critique papers
that question EFH relevance, written by authors that regularly provide expert
testimony defending the ski industry in personal injury lawsuits. We present a
web application that can facilitate jumping injury reduction by calculating EFH
on current and future jumps.

\subsection{History}
\label{sec:hist}
%
Gradual introduction of terrain parks in the 1980's was accompanied by
increased interest in aerial maneuvers and extreme sports participation. Jumps
have proliferated since and are today nearly ubiquitous.  Roughly 95\% of US
ski resorts include terrain parks. Unfortunately, this growth correlates with
injuries. Two early longitudinal studies in the 1980's and early
1990's~\cite{Deibert1998,Furrer1995} already found significant increases in
head injuries and concussions. Between 1993 and 1997 head injuries accompanied
most skiing and snowboarding deaths~\cite{CPSC1999}. Koehle et
al.~\cite{Koehle2002} stated ``[S]eventy-seven percent of spinal
injuries~\cite{Tarazi1999} and 30\% of head injuries~\cite{Fukuda2001} in
snowboarding were a result of jumps.'' Jackson et al.~\cite{Jackson2004}
determined that by 2004 snow skiing replaced football as the second leading
cause of serious head and spinal cord injuries in America.

These early increasing injury assessments persisted. According to
\cite{Russell2014}, ``between 5 and 27\% of skiing and snowboarding injuries
occur[red] in terrain parks
\cite{Bridges2003,Goulet2007,Moffat2009,Greve2009,Brooks2010,Ruedl2013}''.  At
the first Winter Youth Olympic Games over a third of all snowboard half-pipe
and slope-style competitors were injured~\cite{Ruedl2012}.  Epidemiological
research~\cite{Carus2016,Audet2020,Hosaka2020} continues to show that injuries
on terrain park jumps are more likely and more severe than on normal slopes.
Audet et.~al~\cite{Audet2020} provides evidence that skiing or snowboarding in
a terrain park is a risk factor for head, neck, back, and other severe
injuries. Hosaka et. al~\cite{Hosaka2020} concludes that jumping is a main
cause for serious spinal injuries, regardless of skill level, and suggests
that, because spinal injuries' incidence  have not decreased over time, the ski
industry should focus on designing fail-safe jump features to minimize risks of
serious spinal injuries. Similar jump design suggestions have appeared in
peer-reviewed literature for more than a
decade~\cite{Hubbard2009,Swedberg2012,Hubbard2012,McNeil2012,McNeil2012a,Hubbard2015,Levy2015,Petrone2017,Moore2018}.

\section{Methods}
\subsection{Equivalent Fall Height}
\label{sec:efh}
%
\emph{EFH}, a common proxy measure for impact danger in industrial safety
standards, is the weight-specific kinetic energy that must be dissipated on
falling impact from height $h$~\cite{Muller1995,Hubbard2009,Gasser2018}.
Initial potential energy $mgh$ is transformed to kinetic energy available to
injure in non-rotating falls. Injury potential can be reduced by controlling
impact circumstances, e.g. impact cushioning, and body orientation,
configuration, and motion; however this energy must still be dissipated. Larger
EFHs require more elaborate measures to reduce injury; reducing EFH does not.

EFH can be interpreted by the general public. People have an intuitive sense of
danger when faced with potential falls from large heights and a strong
experiential common sense for relating fall height to likelihood of injury.
People sense increasing danger associated with falling from larger heights
because injury severity increases with fall height~\cite{Nau2021}.  Ground,
second, and third floor falls are about 2.6, 5.1, 8.8~\si{\meter},
respectively~\cite{Vish2005}. The German Society for Trauma Surgery's threshold
for trauma team activation is a fall height of 3~\si{\meter}
\cite{PolytraumaGuidelineUpdateGroup2018}. The US Occupational Safety and
Health Administration requires protection for heights over 1.2~\si{\meter} for
general workplace safety~\cite{OSHA2021}.  Chalmers et al.~\cite{Chalmers1996}
argues for 1.5~\si{\meter} maximum fall heights for playground equipment. The
Swiss Council for Accident Prevention makes specific recommendations for EFHs
below 1.5~\si{\meter} for terrain park jumps requiring basic
skills~\cite{Heer2019}.  Even with no standards in Olympic Nordic ski jumps,
typical ``equivalent landing height``~\cite{Gasser2018} is only about
0.5~\si{\meter}.

EFH $h$ of objects is formally defined as
%
\begin{equation}
  h = \frac{v^2}{2g}
  \label{eq:efh_general}
\end{equation}
%
where $v$ is impact velocity and $g$ gravitational acceleration.  Kinetic
energy of objects moving at  velocity $v$  is transformed from potential energy
at height $h$.

Beginning from equation~\ref{eq:efh_general} equivalent fall heights $h$ can be
determined for any surface, i.e., sloped landing profile or shape, after
jumping~\cite{Petrone2017}. The result, neglecting air drag, is
%
\begin{align}
  h = \left[\frac{x^2}{4(x\tan\theta_T - y)\cos^{2}\theta_T} - y\right]
    \sin^{2}
    \left[\tan^{-1}\left(\frac{2y}{x} - \tan\theta_T\right) -
    \tan^{-1}\frac{dy}{dx}\right]
  \label{eq:efh}
\end{align}
%
a function only of takeoff angle $\theta_T$, impact coordinates $(x,y)$
relative to takeoff, and landing surface slope $\frac{dy}{dx}$, but not takeoff
speed~\cite{Petrone2017}. To analyze jumps, one measures Cartesian coordinates
of landing surfaces along jumpers' flight paths and takeoff angles. Slopes
$\frac{dy}{dx}$ are computed from measured coordinates $(x,y)$. Positive
curvatures (concavity) in takeoff ramps tend to cause skiers to rotate
rearwards, inverting them in flight, so they might land in more dangerous body
orientations \cite{Scher2015}, although ramp curvature does not influence EFHs.

\subsection{Software and Online Access}
\label{sec:software}
%
We presented the first version of software for designing ski jumps with a
specified EFH in \cite{Moore2018}. It comprises a general-purpose, extensible,
object-oriented software library with tools for 2D skiing simulation. Using
this code, a web application was developed for interactive jump design. The web
application is designed for a non-technical end-user and operable on any
desktop, tablet, or mobile device supporting a web browser.

We have extended capabilities of this software in version 1.4.0 (March 25,
2021) to assist work described here. New library features automate calculation
of EFH for jump profiles described by a set of Cartesian coordinates.
Additionally, a new ``analysis'' page allows users to upload measured jump
profile coordinates in either a comma separated value or Microsoft Excel
spreadsheet file. Jumps are then analyzed and EFHs are displayed graphically
for interactive user manipulation and viewing.
Figure~\ref{fig:web-app-screenshot} shows the web application with one of the
case study jumps (Salvini v. Ski Lifts Inc.) loaded for analysis and explains
its primary features.
%
\begin{figure}
  \centering
  \includegraphics[width=\columnwidth]{figures/web-app-screenshot.png}
  \caption{\textbf{Screenshot of the ski jump design and analysis web app} To
    use the analysis portion of the app, a user selects ``Ski Jump Analysis''
    from the primary menu [1], uploads a .CSV or .XLS file by dragging it onto
    the screen [3], inspects the input data for accuracy in the table [4], sets
    the takeoff angle [5], runs the analysis by pressing the ``Run Analysis''
    button [6], views results in an interactive plot [2], and downloads
     results by pressing the ``Download EFH'' button [7].}
  \label{fig:web-app-screenshot}
\end{figure}

This software is written in Python and directly depends on popular packages
including Cython~\cite{Behnel2011}, matplotlib~\cite{Hunter2007},
NumPy~\cite{Oliphant2006}, pandas~\cite{McKinney2020}, Plotly \&
Dash~\cite{Plotly2015}, pycvodes~\cite{Dahlgren2018},
SciPy~\cite{Virtanen2020}, SymPy~\cite{Meurer2017}, and xlrd. This software is
open source and licensed under the MIT redistribution license. The source code
is distributed on PyPi~\footnote{\url{https://pypi.org/project/skijumpdesign}}.
Users can submit bug reports, feature requests, code improvements, and
additions at the Gitlab
repository~\footnote{\url{https://gitlab.com/moorepants/skijumpdesign}}. The
software library's documentation is hosted via Read the
Docs~\footnote{\url{https://skijumpdesign.readthedocs.io}}.  Basic examples of
using the library are provided in the documentation and this paper's
supplementary materials. We have also made the web application available for
free use online.~\footnote{ \url{http://www.skijumpdesign.info}}

We do not view the software as the definitive ski jump design and analysis
tool, but rather as a foundation. The tool has been released as open-source so
that refinements and modifications are easy and encouraged. The software was
designed to be extensible and modular. New surface shapes such as different
takeoff ramps are easily added by building upon the basic surface object using
object-oriented programming principles. Similarly, new skier models can be
added that incorporate more complex biomechanical features and actions. We make
use of this flexibility for the web application and for calculations and
visualizations presented in the following section.

\section{Results}
\label{sec:case}
%
In these case studies of American lawsuits, juries ruled for injured
plaintiffs. Negligent jump design and construction contributed significantly to
injuries~\cite{SuperiorCourtSanFranciscoCounty2002,KingCountySuperiorCourt2008}.
Simulations below use methods in \cite{Levy2015}, assuming the same skier mass,
frontal area, and drag coefficient of 75~\si{\kg}, 0.34~\si{\meter\squared},
and 0.821, respectively.

\subsection{Vine v. Bear Valley Ski Company}
\label{sec:vine}
%
In April 2000, Ms.~Vine's lower spine was injured when she landed badly while
jumping on skis at Bear Valley in California. The jump shape
(Fig.~\ref{fig:vine-v-bear-valley}) was a common form called a ``table-top''.
Builders intend that jumpers completely clear the table, landing on down-slopes
near a ``sweet spot''. The upper panel of Fig.~\ref{fig:vine-v-bear-valley}
shows the  measured jump surface from accident investigation. Vine landed short
of the knuckle, defined as the end of the table-top. This table-top was not
flat and horizontal as is typical. Instead it was concave, compounding dangers
of short landings. At the 11~\si{\meter} landing horizontal distance measured
from  takeoff, the surface sloped upwards approximately 5\si{\degree}. Concave
table tops exacerbate detrimental effects of failing to align landing zone
tangents close to jumper flight paths at impact.
%
\begin{figure}
  \centering
  \includegraphics[width=\columnwidth]{figures/vine-v-bear-valley.pdf}
  \caption{\textbf{Bear Valley jump compared to possible safer design}
  Top: Measured landing surface (solid black) and jumper flight paths
  (intermittent black) from measured 30\si{\degree} takeoff angle.  A
  14~\si{\meter\per\second} takeoff speed is used as the design
  speed~\cite{Levy2015} for a comparison jump (solid green) shaped to have
  constant EFH of 1~\si{\meter}.
  Bottom: EFH for both jumps in corresponding colors at 2~\si{\meter}
  intervals. Numbers above bars indicate takeoff speeds required to land at
  that location.
  Intermittent horizontal gray lines indicate increasing relatable fall
  heights: knee collapse, average 1\textsuperscript{st} story fall, and average
  2\textsuperscript{nd} story fall.
  }
  \label{fig:vine-v-bear-valley}
\end{figure}

The lower panel displays EFHs at different landing locations. These are
greatest just short of the knuckle. At the sweet spot, just past the knuckle,
EFH drops precipitously to  about 1~\si{\meter}, although landing in this
narrow region requires jumpers to control takeoff speeds very accurately to
within about 1~\si{\meter\per\second}.  Landing at 11 meters, Vine's EFH was
instead almost 4 meters, equivalent to falling from between one and two
stories~\cite{Vish2005}.  She had also rotated backward in flight, landed on
her lower spine and was paralyzed. Lower EFH would have decreased risk of
injury, due to lower impact forces.

Jumps with smaller EFHs can be created at similar costs. The green jump profile
in the upper panel of Fig.~\ref{fig:vine-v-bear-valley} shows a possible jump
design, see~\cite{Levy2015}, of similar size with similar flight times that
ensures constant (small) EFHs of about 1~\si{\meter}. Interestingly, the convex
shape of this jump is close to the original concave table-top  inverted,
showing that convex landing shapes are critically important for limiting EFHs.
This alternative jump design would have lowered impact forces for landings at
all locations. In 2002, the jury ruled in favor of Ms.~Vine, agreeing that Bear
Valley was responsible for providing unsafe jumps.

\subsection{Salvini v. Ski Lifts Inc.}
\label{sec:salvini}
%
In 2004, Mr.~Salvini attempted a table-top jump on skis in the terrain park of
The Summit at Snoqualmie Ski Resort, in Washington state. Salvini overshot the
intended landing location while traveling at typical skiing
speeds~\cite{Shealy2005}, rotated backward during flight and landed on his
back, ultimately suffering quadriplegia. The jury sided with Mr.~Salvini and he
was awarded a judgment of \$14M.

At his landing location of 30~\si{\meter} the EFH exceeded 10 meters,
approximately a 3-story fall. Figure~\ref{fig:salvini-v-snoqualmie} shows the
measured jump profile from the accident investigation.  For takeoff speeds
over 13~\si{\meter\per\second}, the lower panel shows that the EFH is
over 10~\si{\meter} and growing linearly with larger takeoff speeds.
Severe injury is almost certain in falls this high, especially if landing body
orientation loads the spine, as in this case.
%
\begin{figure}
  \centering
  \includegraphics[width=\columnwidth]{figures/salvini-v-snoqualmie.pdf}
  \caption{\textbf{Snoqualmie jump compared to possible safer design}
  Top: Measured landing surface (solid black) and jumper flight paths
  (intermittent black) for measured 25\si{\degree} takeoff angle. The
  16~\si{\meter\per\second} takeoff speed is used as the design speed for a
  comparison jump (solid green) with constant EFH of
  1~\si{\meter}.
  Bottom: Equivalent fall height for both jumps in corresponding colors at
  2~\si{\meter} intervals. Numbers above bars indicate takeoff speed required
  to land at that location.
  Intermittent horizontal gray lines indicate increasing relatable fall
  heights: knee collapse, average 1\textsuperscript{st} story fall, average
  2\textsuperscript{nd} story fall, and average 3\textsuperscript{rd} story
  fall.
  }
  \label{fig:salvini-v-snoqualmie}
\end{figure}

The upper panel also shows a jump profile (green) designed to have a
1~\si{\meter} EFH for all speeds below 16~\si{\meter\per\second}. This profile
requires significantly more snow than the measured jump but limits EFH to 1 m.
This jump highlights how extreme EFHs can become if jumps are not properly
designed.  Nobody would voluntarily jump out of three story windows, snow or
not, as injuries are clearly likely. Our internal altimeter tells us so, but it
is impossible for recreational skiers to evaluate EFHs simply by looking at
jumps.

These two case studies demonstrate that deficient jump landing shapes have
devastating consequences and that engineering analysis and design based on
well-established laws of mechanics could be used to design jumps that limit
EFHs safely.  Designing jumps this way is based on mechanics elucidated
centuries ago by Isaac Newton and Émilie du Châtelet~\cite{Zinsser2007}, and
fundamental to physics and engineering education. Designing jumps to limit EFHs
unquestionably reduces injury risk by reducing impact energies and associated
forces.

\section{Discussion}
\subsection{Moral Imperative}
\label{sec:moral}
%
``Hold paramount the safety, health and welfare of the
public''~\cite{NSPE2019}, is the first canon of engineering ethics. Ethics is
not a matter of opinion and should not be optional, but rather is the
foundation for engineering. The first canon compels engineers to use their
technical expertise to protect snowsport participants from injuries. Reducing
EFHs cannot increase likelihoods of injuries. Building well designed, safer
jumps is no more laborious than building poorly designed, unsafe jumps. There
is no reason not to control EFHs with ethical design algorithms and software.
Nonetheless ski industries and their insurance companies are reluctant to adopt
and endorse such design methods, choosing instead to invest in litigation
defense rather than technologies for constructing safer jumps. They hire
engineers to profess doubt on the fundamental physics of EFHs during
litigation. Publications cited by the defense in litigation to support these
doubts provide little or nothing for the safety, health, and welfare of the
public, that engineers should hold paramount.

In their book ``Merchants of Doubt''~\cite{Oreskes2010}, Oreskes and Conway
have studied this problem more generally. They show that in numerous industries
over the last 60 years, scientific evidence accumulated that commonly accepted
industrial activities were harmful, to individuals and society. However,
industries had vested interests in continuing practices that were dangerous to
the public, perhaps because operational changes would have led to significant,
short-term costs and inconvenience. Examples carefully described and
analyzed~\cite{Oreskes2010} include using DDT, smoking tobacco, producing acid
rain from coal-fired power plants, causing ozone holes from  CFCs, damaging
health with second-hand tobacco smoke, and changing our climate with CO2
emissions. Rather than using the scientific evidence as a basis for changes in
practice, strategic responses of industries have been to ``emphasize the
controversy among scientists and the need for continued
research''~\cite{Oreskes2010}.

This same strategy is used by some snowsport industries and their defense
experts, who disparage EFHs.  To sow doubt and counter solid, fundamental,
ethical, scientific concepts of jump designs limiting EFH, defense experts
introduce confounding factors to cloud and confuse basic issues. Consider as
evidence three papers \cite{Shealy2010,Shealy2015,Scher2015} co-authored by
well-known ski industry defense experts who have testified for snowsport
resorts and their insurance companies. We do not fundamentally question their
empirical findings but we do question their interpretation of the findings,
namely their conclusion that greater fall height is not a basic indicator of
greater risk of injury.

Shealy et~al.~\cite{Shealy2010} conducted an experimental study attempting to
test the hypothesis that takeoff speed is a predictor of the distance from a
jump take-off to landing. They reached the mechanically impossible conclusions
both that there is ``no statistically significant relationship between takeoff
speed and the distance traveled'' and that ``takeoff speed is not a dominant or
controlling factor (in how far a jumper travels)''~\cite{Shealy2010}. These
conclusions were used to question the soundness of analytical mechanical
modeling of jumper flight used in \cite{Hubbard2009,McNeil2012}.

Some of these same authors later vouched for terrain park jump safety. Using
data held by the National Ski Areas Association (NSAA), Shealy et
al.~\cite{Shealy2015} concluded that their ``hypothesis that jumping features
resulted in an increase risk of injury [was] not \ldots
substantiated.''~\cite{Shealy2015} This is the only study we are aware of with
this conclusion. It is difficult to reconcile it with the voluminous
contradictory research documenting the unique dangers posed by terrain park
jumps in tens of other studies cited both herein and in
\cite{Hubbard2009,Swedberg2012,McNeil2012,McNeil2012a,Hubbard2015,Levy2015,Petrone2017,Moore2018}.
Although NSAA releases yearly totals of resort-related fatalities and
catastrophic injuries, the raw data on which \cite{Shealy2015} was based is not
publicly available, thus making these results unverifiable. The data was
collected from press releases produced by the NSAA~\cite{Shealy2015}, which has
an inherent conflict of interest, thus potentially introducing confounding
bias.

In a third experimental study (N=13) specifically designed ``to evaluate injury
mitigation potential of surfaces limiting EFH''~\cite{Scher2015}, Scher et al.
clearly show that body orientation, i.e. falling directly on one's head (in all
trials), can cause dangerous cervical spine compression loads~\cite{Scher2015},
even at low fall heights. They report on effects of EFH but only test heights
from \SIrange{0.23}{1.52}{\meter},  similar to limitations in \cite{Shealy2010}
by restricting ranges of their independent variables, and ignoring fall heights
known to have caused severe injuries regardless of body orientation. Yet, they
insinuate that EFH has no appreciable effect on injuries.  The title, ``Terrain
Park Jump Design: Would Limiting Equivalent Fall Height Reduce Spinal
Injuries?'' implies that they appear to believe that falling from greater
heights might \emph{not} cause greater injuries. Why propose such mechanically
flawed hypotheses? Sowing doubt on EFH as a basic indicator of risk appears to
be paramount.

Extending the scope of findings is a common mistake, but one that should not be
made by ethical, professional engineers when safety, health, and welfare of the
public is at issue. Fundamental laws cannot be disproved by these kinds of
jumping experiments. If statistical or experimental results seem in conflict
with predictions from classical mechanics, the problems are probably with the
statistical or experimental design or their interpretations, but not
fundamental laws of mechanics. Defending dangerous practices that lead to
injuries helps prolong these practices, which leads to further injuries,
clearly contradictory to ethical engineering. Engineering experts defending ski
industries and their practices could be complicit in continued societal damage,
and in doing harm to the safety, health, and welfare of the public. As Upton
Sinclair wrote ``it is difficult to get a man to understand something when his
salary depends on his not understanding it''~\cite{Sinclair1994}.

It is not evident that these papers~\cite{Shealy2010,Shealy2015,Scher2015}
``hold paramount the safety, health and welfare of the public''. They are
silent on how their findings can be used to reduce injuries. They obscure a
scientifically fundamental, mechanically irrefutable fact that impacting
surfaces at lower normal velocities is safer. They ``create the appearance that
the claims being promoted were scientific''~\cite[page 244]{Oreskes2010}.
Fundamental laws have made mechanics a science. Findings that contradict such
fundamental laws should be carefully scrutinized and review processes accepting
such articles should be questioned.

Organizations also merchandise doubt. A decade ago, NSAA argued~\cite{NSAA2008}
that, because of rider and snow variability, terrain park jump ``standards are
essentially impossible.'' While it is true that the ``virtually \ldots infinite
number of ways that a given feature may be used by an individual \ldots varying
speed, pop, body movement, takeoff stance, angles of approach, the attempting
of different kinds of maneuvers, landing stance, and the type of equipment used
(skis or snowboard) \ldots create a wide variety of experiences for the
users''~\cite{NSAA2008}, none of these in fact preclude EFH analysis and
rational engineering design.  This was shown clearly in
reference~\cite{Hubbard2012} which examined quantitatively the effects of
variations in factors actually involved in the mechanics: takeoff speed, snow
friction, air drag, tail wind, snow melt and jumper pop. These so-called
``uncontrollable factors'' fell into three groups: (1) those for which there is
zero sensitivity, i.e., an uncontrollable factor that makes no difference in
the ability of the designed jump to deliver the designed EFH; (2) those for
which fairly large parameter variations cause only insignificant maximum
deviations in EFH, and (3) those for which the factor can be taken into account
in the design process itself and its larger effect on EFH completely eliminated
in the unsafe direction. The allegation that design of limited EFH surfaces is
prevented by the complexity of the problem and by the large number and types of
parameter variations away from nominal is false; in fact the allegation is just
more merchandised doubt.

In snowsport injury cases, testifying for injured plaintiffs and testifying
defending corporations are not ethically equivalent. The former attempts to
address problems that cause injuries, holding paramount the public's safety,
health, and welfare. The latter attempts to defend practices that might have
contributed to the injury, to limit financial losses of corporations. The idiom
``two sides to every question'', is not appropriate in science and
engineering~\cite[page 268]{Oreskes2010}.

Engineers whose scholarly work ignores engineering's first canon of ethics in
favor of merchandising doubt can diminish the scientific integrity of
engineering journals and engineering conferences. Journal editors should
recognize submissions primarily intending to cast doubt on good science and
engineering for what they are, tools of insurance companies for defending civil
suits, and reject these submissions. Papers whose findings help to perpetuate
dangerous practices for the short-term financial benefit of industry and which
apparently do nothing for the safety, health, and welfare of the public, are
unethical and do not belong in engineering journals or conference proceedings.

\subsection{What Can Be Done?}
\label{sec:action}
%
Absolutely, the most important change will be to incorporate rigorous, rational
processes and scientific principles that consider mechanical impact safety into
designing freestyle jumps.  At present a large fraction of, if not most, jumps
in the USA are created in a formulaic way using two straight lines, a
horizontal deck (tabletop) and nearly constant-slope landing region, linked by
a curved knuckle. This design philosophy is recommended in the instructions
provided by the NSAA~\cite{NSAA2015} and is presumably followed by their
members. Although such jumps are simple and thus easy to design, research has
shown that jumps with bi-linear geometry have generally poor EFH
behavior~\cite{Swedberg2012}, i.e. that they can have low EFH only in a small
region just past the knuckle (called the ``sweet spot''). In a recent version
of their freestyle terrain park notebook~\cite{NSAA2015}, the jump landing area
is even termed the ``landing plane'' because it is envisioned to be planar.
There is no reference to any concept such as EFH or similar measure of impact
or its effect on safety because the NSAA's strategy is to put the
responsibility for safety fully on the jumper. There is no quantitative
consideration of jump impact safety (e.g. from the point of view of EFH) beyond
the experience of the designer. The skiing industry continues to resist more
scientifically-based rational approaches to design, in spite of the fact that
computer aided design (and even computer-assisted fabrication and maintenance)
of snow park jumps (see Figure~\ref{fig:prinoth}) has been available from snow
groomer manufacturers for over 5 years~\cite{Muigg2019}. The 2015 NSAA
reference in~\cite{NSAA2015} still contained the statement that ``Standards are
essentially impossible~\ldots''.
%
\begin{figure}
  \centering
  \includegraphics[width=\columnwidth]{figures/prinoth.png}
  \caption{\textbf{Commercial availability of computer-aided design and
    computer-controlled fabrication of snow park surfaces began as early as
    2016.} The right panel shows Prinoth's computer generated 3-D jump landing
    surface with their family of simulated jumper paths, even ones outside the
    central vertical bisecting plane, with the landing surface colored
    corresponding to the EFH incurred by the jumper at that landing point. The
    left panel shows a computer-controlled snow groomer fitted with two GNSS
    receivers that allow real time measurement of their position to an accuracy
    of about 2~\si{\centi\meter}, calculation of the yaw and roll of the
    groomer blade, and precise closed loop control of the snow addition and
    removal process.  Images courtesy of Prinoth, supplier of snow groomers for
  the winter Olympics in China 2022.}
  \label{fig:prinoth}
\end{figure}

Once the jump surface has been designed, the next most important change is to
build accurately what was designed. Presently a dominant fraction of jumps are
simply fabricated by groomer operators, based on perhaps a few measurements of
distances and slopes (deck length, takeoff angle, landing region angle and
length) during the process. But the design concepts are overly simple and do
not incorporate or address quantitative indicators of safety such as EFH. The
introduction of computer controlled grooming (see Figure~\ref{fig:prinoth}),
similar to computer aided manufacturing (CAM) and machining (CNC), will
facilitate construction of more complex designed shapes precisely and
accurately to within ten centimeters. These would include the non-trivial
constant EFH surfaces provided by our online ski jump design software that
limit landing impulses to acceptable levels.

Every jumper (and parent of young jumpers) should be able to confirm that a
jump is not unsafe before trying it. Appropriate inspection, evaluation,
correction, and maintenance of existing jumps, and the design and construction
of safer new jumps should be promoted.  Postings should be required and include
EFHs, the certified inspectors name, and when last inspected and maintained.
Inspections should be frequent enough to ensure that jumps meet safe design
standards, particularly regarding takeoffs and starting points to prevent
inadvertent inversions due to take off ramp curvature. Standards need to be
developed that limit EFHs in collaboration between industry and research
engineers to design, build, inspect, maintain, and post safer jumps. An example
of first steps in this area is a terrain park safety guide by the Swiss Council
for Accident Prevention~\cite{Heer2019}.

To complement standards, certification programs are needed for jump building,
inspection, and maintenance. ASTM, an American organization for a wide variety
of consensus standards, provides a historical example of a successful
certification program. ASTM Committee F27 was created in 1982 for skiing safety
and began to develop ski binding standards.  Proponents were led by orthopedic
surgeons and academic researchers~\cite{Bahniuk1996}. Industry argued that
standards were unattainable because release value measurement was impossible by
ski shops, just as industry now makes similar arguments about
jumps~\cite{NSAA2015}. Nevertheless certifications and inspection standards for
bindings were developed, which led to fewer lower-extremity equipment-related
injuries~\cite{Bahniuk1996}.

Now however no medical professionals and almost no academics remain in F27.
Efforts to create similar standards for terrain park ski jumps began in F27
more than a decade ago~\cite{SAM2011}, yet no standards have yet resulted with
any appearance of increasing safety for the public. The US skiing industry,
aided by the NSAA, has been successful in delaying the implementation of
standards.

In parallel with standards development, assessing and possibly reshaping
existing jumps to eliminate dangerous EFHs should be a straightforward route
for ski resorts to proactively increase terrain park safety. Accurate enough
measurements of existing surfaces can occur even with simple tools, e.g. tape
measure and digital level, and consume relatively little time and effort per
jump (see supplementary materials for details). Calculation and visualization
of EFHs from these measurements can take some time without a computational
program for calculating EFHs from hill profiles. The user-friendly,
freely-accessible open-source online web application tool that we have made
available for jump designers and builders has almost instantaneous calculation
and visualization steps, solving this problem.

With this software, jump builders can add safety assessment to their toolbox,
even accessing it from a smartphone or tablet on hills.  We see no reason that
this basic assessment should not be part of every jump construction process.
The only ethical decision is to adopt these methods; saving even one person
from a life of paralysis, or even death, must be worth the relatively minor
inconvenience of shaping jumps using the methods in reference~\cite{Levy2015}.

\section{Conclusion}
\label{sec:conc}
%
There are, of course, more factors than jump takeoff and landing profiles that
contribute to injuries on terrain park jumps. Yet normal impact velocity can be
easily controlled with a properly designed and fabricated jump. There is no
evidence that decreasing designed EFH increases injuries in falls; injuries can
only decrease. Thus we see no reason not to adopt constant low values of EFH
for public-use jump designs. Builders of jumps that are not designed as
forgiving environments are negligent. Public safety must be held paramount to
short-term return-on-investment.

The methods implemented in the software illustrated in
Section~\ref{sec:software} provide a starting point for realizing EFH-conscious
designs in terrain parks. We hope to see the design and analysis adopted by
commercial grooming equipment manufacturers so that safety is made integral to
jump design. Our software can grow and evolve through contributions from other
researchers to incorporate many other nuances of injury prevention. We also see
the methods providing a structure for standards development. And minimally, we
see the software as an immediately usable tool for jump fabricators in the
field.

\section*{Acknowledgements}
We thank Rado Dukalski for feedback on the web application and Yumiko
Henneberry, Lyn Taylor, Andy Ruina, and Ton van den Bogert for feedback on the
manuscript.

\section*{Declarations}
\begin{description}
  \item[Funding] Not applicable
  \item[Conflict of interest] MH served as a plaintiff's expert witness in the
    two case studies discussed above and in numerous other similar cases since.
    CB testifies occasionally on behalf of plaintiffs in ski and snowboard
    injury cases. He has collaborated with Shealy and C.~D. Mote, Jr., Sher's
    doctoral advisor, on ski safety research, has participated in ASTM F27
    since the 1980s on standards for bindings, boots, and skis, and holds
    patents on ski and snowboard binding designs intended to reduce injuries.
  \item[Availability of data and material] All data is available at
    \url{https://gitlab.com/moorepants/skijumpdesign} and
    \url{https://gitlab.com/mechmotum/ski-jump-analysis-paper}.
  \item[Code availability] The skijumpdesign version 1.4.0 source code is
    archived at \url{https://doi.org/10.5281/zenodo.4637076}. Additionally, it
    and the paper's source code is available at
    \url{https://gitlab.com/moorepants/skijumpdesign} and
    \url{https://gitlab.com/mechmotum/ski-jump-analysis-paper}.
  \item[Author's contributions] JM and MH contributed to the study conception
    and design. Material preparation, data collection and analysis were
    performed by JM and MH. The first draft of the manuscript was written by
    JM, BC, MH, and CB. MH and CB were primarily responsible for drafting the
    parts on merchandising doubt and ethics, respectively. All authors read and
    approved the final manuscript. BC and JM wrote the accompanying software.
  \item[Ethics approval] Not applicable
  \item[Consent for publication] JM, BC, MH, and CB consent for publication.
\end{description}

\bibliographystyle{ieeetr}
\bibliography{references}   % name your BibTeX data base

\end{document}
